\documentclass[a4paper,12pt,fleqn]{scrartcl}
\usepackage[l2tabu,orthodox]{nag}% Old habits die hard. All the same, there are commands, classes and packages which are outdated and superseded. nag provides routines to warn the user about the use of those.
\usepackage[all,error]{onlyamsmath}% Error on deprecated math commands like $$ $$.
\usepackage{fixltx2e}
\usepackage[strict=true]{csquotes}
\usepackage{listings}
\lstset{frame=single,framerule=0pt,language={C},showstringspaces=false,numbers=left,columns=fullflexible}
%\usepackage{color}
\usepackage[usenames, dvipsnames]{color}
\usepackage{2111defs2,2111theorems}
\usepackage{nicefrac}

\newcommand{\var}{\textrm{$var$ }}
\newcommand{\singlequote}[1]{`#1'}
\newcommand{\myjustification}[2][\Equiv]{{}#1{} &{\bcolor\langle\text{#2}\rangle}\\}
\newcommand{\remark}[1]{{\sffamily\color{blue}{#1}}}
\newcommand{\myImplies}[2]{\color{red}#1\color{blue}\Implies\color{red}#2\color{black}}
\newcommand{\pre}{\mathit{pre}}
\newcommand{\post}{\mathit{post}}
\newcommand{\myCode}[1]{\mathbf{\color{ProcessBlue}#1}}
\newcommand{\Dict}{\mathbf{Dict}}
\newcommand{\DictA}{\mathbf{DictA}}
\newcommand{\domt}{\mathbf{dom}t}
\newcommand{\domto}{\mathbf{dom}t_{0}}
\newcommand{\myRefines}[2]{\color{red}{#1}\color{black}\isrefinedby\color{red}{#2}}

\title{Assignment 3 - Emirps}
\date{COMP2111 18s1}
\author{Rahil Agrawal z5165505\\Aditya Karia z5163287}

\begin{document}
\pagenumbering{gobble}
\maketitle
\pagenumbering{Roman}

\section{Task 1 -Syntactic Data Type Dict}
\label{sec:task-1}
We define a syntactic data type $\Dict$ that encapsulates a dictionary set $W$ as follows :
\begin{center}$\Dict = (init^{Dict} , (W,x : [pre_{j}^{Dict}, post_{j}^{Dict}]_{j\in J}))$\end{center}
which consists of an  initialisation predicate \color{blue}$init^{Dict} = (W = \{ \})$ \color{black} and the following operations:\\
$proc$ $addword^{Dict}$(value w) . $W : [True, \color{blue}W = W_{0} \cup w\color{black}]$\\
$proc$ $checkword^{Dict}$(value W, value w, result b) . $b : [True, \color{blue}b = (W = \{ \})\color{black}]$\\
$proc$ $delword^{Dict}$(value w) . $W : [\color{blue}W\neq \{ \}, W = W_{0} \setminus {w}\color{black}]$\\
\section{Task 2 - Refinement to DictA}
\label{sec:task-2}
We’re refining this to a data type DictA where we replace W with a Trie t. From Ass3 2018 S1 Specification, "A $trie$ $domain$ is a prefixclosed finite subset of $L^*$. A $trie$ is a function from a trie domain to Booleans. Given a trie $t$ we write \textbf{dom}$t$ for its trie domain. Let $T$ be the set of all tries."\\ 
The correspondance between the two data types is captured by the function $f : T \mapsto P(L^*)$ given by :
\begin{center}$f(t) = \{w\in L^* \mid w \in \textbf{dom}t \And t(w)=1\}$\end{center}
Also, from Ass3 2018 S1 Specification, "Formally, we write $v\leq w$ if word $v\in L^*$ is a prefix of $w\in L^*$, i.e., $\Exi{v'\in L^*}{vv' = w}$. We write B for {0, 1} where 0 represents ‘false’ and 1 ‘true’."\\
We define our 
With the aforementioned facts in mind, we define a concrete syntactic data type $\DictA$ that encapsulates a trie $t$ as follows :
\begin{center}$\DictA = (init^{DictA} , (t,x : [pre_{j}^{DictA}, post_{j}^{DictA}]_{j\in J}))$\end{center}
which consists of an  initialisation predicate \color{blue}$init^{DictA} = (t = \{ \})$ \color{black} and the following operations:\\
$proc$ $addword^{DictA}$(value w) . $t : [True, \color{blue}\post(addword^{DictA})\color{black}]$ where
\begin{gather*}
\color{blue}\post(addword^{DictA})\color{black} = \domt = \domto \cup \{v\in L^* \mid v\leq w\} \And t = t_{0}\text{ }\cup \\
\qquad\qquad\qquad\qquad\qquad \{v\in L^* \mid v\in\domt\text{ } \And v<w \And t_{0}(v) = 1 \Implies t(v) = 1\\
\qquad\qquad\qquad\qquad\qquad\qquad\qquad\qquad\qquad\qquad v<w \And t_{0}(v) = 0 \Implies t(v) = 0\\
\qquad\qquad\qquad\qquad\qquad\qquad\qquad\qquad\qquad\qquad v=w \Implies t(v) = 1\}
\end{gather*}
$proc$ $checkword^{DictA}$(value t, value w, result b) . $b : [True, \color{blue}b = (t(w) = 1)\color{black}]$\\
$proc$ $delword^{DictA}$(value w) . $t : [\color{blue}t\neq \{ \}, t(w) = 0\color{black}]$\\

That this indeed is a refinement requires checking the relevant proof obligations:
\begin{gather}
\color{blue}init^{DictA}\color{black} \Implies \color{red}init^{Dict}\subst {f(t)}W\\
\color{red}pre_j^{Dict}\subst {f(\color{blue}t\color{red})}W\color{black} \Implies \color{blue}pre_j^{DictA}\color{black} , for\text{ } j\in J\\
\color{red}pre_j^{Dict}\subst {f(\color{blue}t_0\color{red}), x_0}{W, x}\color{black} \And \color{blue}post_j^{DictA}\color{black} \Implies \color{red}post_j^{Dict}\subst {f(\color{blue}t_0\color{red}), f(\color{blue}t\color{red})}{W_0, W}\color{black}, for\text{ } j\in J
\end{gather}
We begin with (1).
\begin{align*}
&\color{blue}init^{DictA}\subst {f(\color{blue}t_0\color{red}), x_0}{W, x}\color{black} = f(t) \neq \{ \}\\
%
\myjustification[\Implies]{def. of f}
&f(t) = \{ \} \Implies\\
%
\myjustification[\Implies]{def. of Dict}
&\color{red}init^{Dict}\subst {f(t)}W\\
\end{align*}
Condition (2) is only required to be proven when the concrete pre-condition is non-trivial (not True). This is only the case in delword.
\begin{align*}
&\color{red}pre_{delword}^{Dict}\subst {f(\color{blue}t_0\color{red}), x_0}{W, x}\color{black} = f(t) \neq \{ \}\\
%
\myjustification[\Implies]{def. of f}
&w\in \domt \And t(w) = 1\\
%
\myjustification[\Implies]{def. of trie}
&t \neq \{ \}\\
%
\myjustification[\Implies]{def. of $pre_{delword}^{DictA}$}
&\color{blue}pre_{delword}^{DictA}
\end{align*} 
Finally, condition (4) needs to be checked for all operations.\\
For addword, we prove
\begin{align*}
&\color{red}pre_{addword}^{Dict}\subst {f(t)}W \And \color{blue}post_{addword}^{DictA}\color{black} = True \And \domt = \domto \cup \{v\in L^* \mid v\leq w\} \And t = t_{0}\text{ }\cup \\
&\qquad\qquad\qquad\qquad\qquad \{v\in L^* \mid v\in\domt\text{ } \And v<w \And t_{0}(v) = 1 \Implies t(v) = 1\\
&\qquad\qquad\qquad\qquad\qquad\qquad\qquad\qquad\qquad\qquad v<w \And t_{0}(v) = 0 \Implies t(v) = 0\\
&\qquad\qquad\qquad\qquad\qquad\qquad\qquad\qquad\qquad\qquad v=w \Implies t(v) = 1\}\\
%
\myjustification[\Implies]{Trie is the same as old trie but we added the prefixes of w}
\myjustification[]{without changing their old mappings}
\myjustification[]{New prefixes are mapped to 0 and w is mapped to 1.}
\myjustification[]{This means we added a word w if it did not already exist, def. of trie and f}
&\color{black} f(t) = f(t_0) \cup {w}\\
%
\myjustification[\Implies]{Definition of $addword^{Dict}$}
&\color{red}\post_{addword^{Dict}}\subst {f(t_0), f(t)}{W_0, W}\\
\end{align*}\\
For checkword, we prove
\begin{align*}
&\color{red}pre_{checkword}^{Dict}\subst {f(t)}W \And \color{blue}post_{checkword}^{DictA}\color{black} = True \And b = (t(w) = 1)	 \\
%
\myjustification[\Implies]{b is 1 if w is in $\domt$ and trie maps w to 1, 0 otherwise. def. of trie.}
\myjustification[\Implies]{b = 1 means w is in our set of words. def of f}
&b = (w \in f(t))\\
%
\myjustification[\Implies]{def. of $checkword^{Dict}$}
&\color{red}\post_{checkword^{Dict}}\subst {f(t_0), f(t)}{W_0, W}
\end{align*}\\
For delword, we prove
\begin{align*}
&\color{red}pre_{delword}^{Dict}\subst {f(t)}W \And \color{blue}post_{delword}^{DictA}\color{black} = W \neq \{ \} \And b = (t(w) = 0)\\
%
\myjustification[\Implies]{def of f}
&w \notin f(t)\\
%
\myjustification[\Implies]{Clearly}
&f(t) = f(t_0) \setminus {w}\\
%
\myjustification[\Implies]{def. of $delword^{Dict}$}
&\color{red}\post_{delword^{Dict}}\subst {f(t_0), f(t)}{W_0, W}\
\end{align*}\\
\pagebreak
\section{Task 3 - Derivation}
\label{sec:task-3}
\begin{align*}
  &\PROC~FunctionName(\VALUE~n, \RESULT~r)\cdot{}	
  \nt{n,r,x:\left[
    \begin{array}{l}
      Pre, Post
    \end{array}
  \right]}{(1)}\\
% 
  \lrefstep{(1)}
  {\textbf{Rule}}
  {
  \nt{r,x:\left[
    \begin{array}{l}
      New Pre, New Post
    \end{array}
  \right]}{(2)}}\\
%
  \lrefstep{(2)}
  {\textbf{seq}}
  {
  \nt{i,x,r:\left[
    \begin{array}{l}
      Pre, Blah
    \end{array}
  \right]}{(3)};\\
  &\nt{i,x:\left[
    \begin{array}{l}
      Blah, Post
    \end{array}
  \right]}{(4)}}\\
%
  \refstep{\textbf{ass - \color{blue}(1)}}
  {\myCode{i\Ass 1}\\
  &\myCode{x\Ass 13}}\\
%
  \refstep{\textbf{seq}}
  {
  \nt{s,x:\left[
    \begin{array}{l}
      \pre(16), s=0 \And x>0
    \end{array}
  \right]}{(18)};\\
  &\nt{a,i,r,s,x: \left[
    \begin{array}{l}
      x>0 \And s=0, \post(16)
    \end{array}
  \right]}{(19)}}\\
%
  \lrefstep{(5)}
  {\textbf{while}}
  {
  \myCode{\WHILE~j\neq r~\DO}\\
  &\qquad \nt{r, j:\left[
    \begin{array}{l}
      Inv_{2}\And j\neq r, Inv_{2}
    \end{array}
  \right]}{(8)}\\
  &\myCode{\OD;}}\\
%
  \lrefstep{(9)}{\textbf{if}}
  {\myCode{\IF~\text{Gaurd}}\\
  &\myCode{\THEN}~\nt{a:[\text{Gaurd}\And\pre(9),\post(9)]}{(11)}\\
  &\myCode{\ELSE}~\nt{a:[\text{Not Gaurd}\And\pre(9),\post(9)]}{(12)}\\
  &\myCode{\FI;}}\\
\end{align*}
We gather the code for the procedure body of blah:
\begin{gather*}
  \mathbf{blah(r,n):}\\
  \qquad \var i\Ass 1;\\
  \qquad \var x\Ass 13;\\
  \qquad r\Ass 13;\\
  \qquad \WHILE~j\neq r~\DO\\
  \qquad \qquad x\Ass {x+1};\\
  \qquad \qquad \var a\Ass 1;\\
  \qquad \qquad isPrime(x,a);\\
  \qquad \qquad \IF~a=1~\THEN\\
  \qquad \qquad \qquad \var s\Ass 0;\\
  \qquad \qquad \qquad reversen(x,s);\\
  \qquad \qquad \qquad \var b\Ass 1;\\
  \qquad \qquad \qquad isPrime(s,b);\\
  \qquad \qquad \qquad \IF~b=1\And s\neq x~\THEN\\
  \qquad \qquad \qquad \qquad i\Ass {i+1};\\
  \qquad \qquad \qquad \qquad r\Ass x;\\
  \qquad \OD;
\end{gather*}
Also, we gather the code for the procedure body of blah:
\begin{gather*}
  \mathbf{isPrime(r,j):}\\
  \qquad \var j\Ass 2;\\
  \qquad \WHILE~j\neq r~\DO\\
  \qquad \qquad \IF~(r \textrm{ mod }j)=0~\THEN\\
  \qquad \qquad \qquad a\Ass 0;\\
  \qquad \qquad j\Ass {j+1};\\
  \qquad \OD;\\
\end{gather*}
We have derived our code. However we need to prove \textbf{\color{blue}some }\color{black} refinements.

\subsection{\color{blue}Implication 1\color{black}: $\myRefines{[BLAH, BLAH]}{BLAH}$}
\begin{align*}
&\text{To prove:} BLAH \Implies BLAH \\\\
%
&\text{Proof:}\\
%
&LHS = BLAH\\
%
\myjustification[\Implies]{BLAH}
&BLAH\\
%
\myjustification[\Implies]{BLAH}
&BLAH\\
%
\myjustification[\Implies]{BLAH}
&BLAH\\
%
\myjustification[\Implies]{BLAH}
&BLAH\\
%
\myjustification[\Implies]{BLAH}
&RHS\\
\end{align*}

\subsection{\color{blue}Implication 2\color{black}: $\myRefines{[BLAH, BLAH]}{BLAH}$}
\begin{align*}
&\text{To prove:} BLAH \Implies BLAH \\\\
%
&\text{Proof:}\\
%
&LHS = BLAH\\
%
\myjustification[\Implies]{BLAH}
&BLAH\\
%
\myjustification[\Implies]{BLAH}
&BLAH\\
%
\myjustification[\Implies]{BLAH}
&BLAH\\
%
\myjustification[\Implies]{BLAH}
&BLAH\\
%
\myjustification[\Implies]{BLAH}
&RHS\\
\end{align*}

\section{Task 4 - C Code}
\label{sec:task-4}
\lstinputlisting{dict.c}
\end{document}
